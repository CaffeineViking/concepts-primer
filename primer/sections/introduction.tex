\section{Introduction} \label{sec:introduction}

Right now \Cpp\ is experiencing a \emph{renaissance}, fueled by new language features and an extended standard library introduced with \Cpp11, \Cpp14, and \Cpp17. It's a different beast than \Cpp98 (want to feel old?: that's two decades ago!), while still keeping to its core philosophies: the \emph{zero-overhead principle}, having a \emph{simple \& direct mapping to hardware}, and to be completely \emph{multi-paradigm}. The language has become more powerful, but also simpler, since a lot of the rough edges in \Cpp98 have been removed, or replaced, with modern variants. Interest in \Cpp\ has surged again, and with \Cpp20 coming soon\texttrademark, doubly so.

\Cpp20 is expected to bring features such as \emph{Contracts}, \emph{Coroutines}, \emph{Ranges}, and \emph{Concepts} to the language. Each one of these features deserves their own version of this primer, a self-contained, textbook-style way to introduce people to these features. It should motivate why this feature is needed, what problems it's trying to solve, and present previous alternatives already in the language. After explaining where these existing methods fall short, the feature is shown, demonstrating in which cases it's better than the alternatives. Finally, the featured syntax is presented by using short (to the point!) practical examples.

In this primer we'll be looking at Concepts, a way to \emph{constrain templates}, which leads to: \emph{clearer template errors}, a way to \emph{overload based on constraints}, and more \emph{explicitly defined function template interfaces} by using requirements. The goals of this primer are to teach you how to \emph{apply constraints} to template parameters by using the \emph{requires clause}, and how to \emph{define} your own set of \emph{requirements} by using the various forms of \emph{requires expressions}. You'll then see that we can compose requirements together to form useful \emph{concepts}, that can be used together with \emph{terse syntax}, for writing less verbose generic code.

The primer has the following structure: in \cref{sec:generic_programming} we present the state of \emph{generic programming} with \emph{unconstrained templates}, and some of its problems. We then show some solutions (e.g. type traits), where they fall short, and where concepts may help. In \cref{sec:concepts} we introduce concepts, and show how to use \emph{requires clauses} and \emph{requires expressions} when constraining templates. The three major terse syntax proposals are then presented in \cref{sec:terse_syntax}, and is followed by \cref{sec:standard_library_concepts} with an overview of the \emph{concepts and ranges library}. Finally, we wrap things up in \cref{sec:summary}, and see what the future might hold.

\subsection*{Building Examples}

Alongside this document you'll find plenty of examples on how to use concepts, and you'll be happy to know \emph{almost} all of them compile out-of-the-box when using GCC with the \texttt{-fconcepts} flag! If you are a boring person, you can clone the \href{https://github.com/CaffeineViking/concepts-primer}{repository}, or, if you're feeling adventurous, \emph{unzip} this PDF with:

\begin{lstlisting}[language=bash, morekeywords={unzip, make},
                                  deletekeywords={test}]
unzip concepts-primer.pdf
cd examples
make gcc-test && make -j8
\end{lstlisting}
